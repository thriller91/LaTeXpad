\documentclass[twocolumn, 10pt]{article}
\usepackage[francais]{babel}
\usepackage[T1]{fontenc}
\usepackage[utf8]{inputenc}
\usepackage[top=2cm, bottom=2.5cm, left=2.5cm, right=2cm]{geometry}
\usepackage[usenames,dvipsnames]{color}
\usepackage{graphicx}
\usepackage{amsmath,amssymb,amsfonts}
\usepackage{float}
\usepackage{caption}

\usepackage{xspace}
%les trois commandes suivantes sont utiles pour écrires des kets
\newcommand{\ket}[1]{\ensuremath{|#1\rangle}\xspace}
\newcommand{\bra}[1]{\ensuremath{\langle #1|}\xspace}
\newcommand{\psh}[2]{\ensuremath{\langle #1|#2\rangle}\xspace}
\newcommand{\epr}[0]{\psi_{\scriptscriptstyle \mathrm{EPR}}}

\title{ xz \\ Photons intriqués et inégalités de Bell}
\author{Hermine de Château-Thierry \and{Zubair Iftikhar} \and{Hélène Spisser}}
\date{26 octobre 2012 \\ \vspace{1em} \texttt{Groupe \no{}1 \hspace{2em} Trinôme \no{}12}}


\begin{document}

\maketitle
\begin{abstract}
{{\itshape \bfseries
    La mécanique quantique c'est trop bien!
    (version diminuée $\wedge \wedge$)
}}
\end{abstract}



\section{Préparation du TP}

\subsection{\'Etat de polarisation d'un photon}

\begin{enumerate} 

    \item \par $\bullet$ L'état de polarisation d'un photon ayant une polarisation faisant un angle $\alpha$ avec la verticale est de la forme suivante:
        \[\ket{\psi} = \cos{\alpha} \ket{V} +  \sin{\alpha} \ket{H}\] 
La probabilité de mesurer l'état $\ket{V}$ pour un tel photon est donnée par $\cos^2{\alpha}$.

    \par $\bullet$ Un photon polarisé circulairement se décrit de la façon suivante:
\[\ket{\psi} =  \frac{1}{\sqrt{2}} ( i \ket{V} +  \ket{H} )\]
La probabilité de mesurer l'état $\ket{V}$ pour un tel photon est donnée par $1/2$.

    \item \par $\bullet$ Le photon décrit par $\ket{\psi} = \ket{V}$ s'écrit de la façon suivante dans la base inclinée: 
\[\ket{\psi} = \cos{\alpha} \ket{V_{\alpha}} +  \sin{\alpha} \ket{H_{\alpha}}\]
La probabilité de le mesurer dans l'état $\ket{V_{\alpha}}$ est donnée par $P_{V_{\alpha}} = \cos^2{\alpha}$.
    \par $\bullet$ Le photon décrit par $\ket{\psi} = \ket{H}$ s'écrit de la façon suivante dans la base inclinée: 
\[\ket{\psi} = -\sin{\alpha} \ket{V_{\alpha}} +  \cos{\alpha} \ket{H_{\alpha}}\]
La probabilité de le mesurer dans l'état $\ket{V_{\alpha}}$ est donnée par $P_{V_{\alpha}} = \sin^2{\alpha}$.

\end{enumerate}

\subsection{Paires de photons intriqués en polarisation}

\begin{enumerate}
\setcounter{enumi}{2}
    \item Par définition, $P_{V_{\alpha}} = | \psh{V_{\alpha}}{\epr} |^2$. Ainsi, 
\[P_{V_{\alpha}} = \left|\frac{1}{\sqrt{2}} ( \psh{V_{\alpha}}{V,V} + \psh{V_{\alpha}}{H,H} ) \right|^2 \]
On ne s'intéresse ici qu'au photon \no{}1. Donc 
\[P_{V_{\alpha}} = \left|\frac{1}{\sqrt{2}} ( \psh{V_{\alpha}}{V} + \psh{V_{\alpha}}{H} ) \right|^2 \]
D'où $P_{V_{\alpha}} = 1/2$ quelque soit $\alpha$, d'après les probabilités calculées à la question précédente.

    \item On a $P(V_{\alpha}, V_{\beta}) = | \psh{V_{\alpha}, V_{\beta}}{\psi_{\textstyle EPR}} |^2$, en développant, il vient:
\[P(V_{\alpha}, V_{\beta}) =  \left|\frac{1}{\sqrt{2}} ( \psh{V_{\alpha}, V_{\beta}}{V,V} + \psh{V_{\alpha}, V_{\beta}}{H,H} ) \right|^2 \]
\[P(V_{\alpha}, V_{\beta}) =  \left|\frac{1}{\sqrt{2}} ( \cos{\alpha} \cos {\beta} + \sin{\alpha} \sin{\beta} ) \right|^2 \]
\[P(V_{\alpha}, V_{\beta}) =  \frac{1}{2} \cos^2(\alpha - \beta) \]

    \item Montrons que le ket $ 1/\sqrt{2} (\ket{V_{\alpha}, V_{\alpha}} + \ket{H_{\alpha}, H_{\alpha}})$ correspond à l'état EPR quelque soit l'angle $\alpha$. En fait, on a :
\[\ket{V,V} = \cos{\alpha} \ket{V_{\alpha},V} +  \sin{\alpha} \ket{H_{\alpha},V} +...\]


\end{enumerate}


\section{Partie expérimentale}

\par Nous allons à présent essayer de fabriquer cet état $\ket{\epr}$ très particulier. Nous utiliserons un couple de cristaux non-linéaires pour générer des photons intriqués. Un dispositif de détection de coïncidences nous permettra de compter paires de photons intriquées.

\subsection{Modules de détection de photons uniques}

\par La détection des photons se fait par des compteurs basés sur des photodiodes à avalanches très sensibles.

\subsection{Compteurs et détecteur de coïncidences}

\par Les signaux des modules de detection sont envoyés à une carte FPGA sur laquelle on peut régler le temps de la fenêtre de détection à l'aide de commutateurs. 

\begin{enumerate}
    \item \par Le principe de la détection de coïcidence est que lorsqu'une paire de photon intriquées arrive, elle se projète dans un certain état juste avant d'arriver sur l'un des détecteurs. Les deux parties de la paire ayant la même projection, les deux détecteurs vont détecter un `click' quasi-simultané.
    \par Ce `click' sera communiqué à l'électronique de comptage, et si les deux `click' se sont produit dans la même fenêtre de coïncidence, le compteur sera incrémenté. Il faut donc que les `click' provenant des détecteurs --- et dus aux paires de photons intriquées --- arrivent le plus simultanément possible sur la carte électronique. Il faut donc placer des câbles coaxiaux de la même longueur.
\end{enumerate}

\par Lorsque le système de comptage est en fonctionnement, on mesure au \textit{maximum} $0.5\ 10^{6}$ coups par seconde sur les détecteurs lorsque la lumière est allumée.

\subsection{Vérification du fonctionnement du détecteur de coïncidences: coïncidences fortuites}

\par Toutes les coïncidences ne sont pas associées à l'arrivée d'une paire de photons intriqués. D'ailleurs la source n'a pas encore été mise en route. Les coïncidences mesurées dans ces conditions sont caractériques de la pollution lumineuse et de la largeur de la fenêtre de mesure.

\begin{enumerate}
\setcounter{enumi}{1}
    \item \par Le nombre de coïncidences qu'on mesure au total pendant une fenêtre temporelle $\tau_f$ est appellé $n_f$. Il est égal au produit du nombre de photons détectés sur la voie A et sur la voie B pendant cette même fenêtre (\textit{i.e.} $(n_A \tau_f) \times (n_B \tau_f)$).
    \par On obtient alors $n_f = n_A n_B \tau_f$ en simplifiant par $\tau_f$. On voit qu'il y a d'autant plusde coïncidences fortuites que les nombres de photons détectés sur les voies A et B sont élevés. Et on note aussi qu'il y en a moins si on choisit une fenêtre de coïncidence plus étroite.

\item \par On relève les nombres de coïncidences fortuites en fonction de la largeur de la fenêtre de détection \footnote{Nous avons uniquement vérifié la linéarité du nombre de coïncidences fortuites avec la taille de la fenêtre. Nous n'avons pas déduit la largeur de la fênetre à partir du nombre de coups détectés sur chaque voie.}, les données sont rassemblées dans le tableau \ref{fortuites}.
\begin{table}[H]
\centering
    \begin{tabular}{||c|c|c||}
    \hline
    SW16 & SW17 & coïncidences fortuites \\ \hline \hline
    0 & 0 & 10000 \\ \hline %VALEURS A REMPLACER
    1 & 0 & 3000 \\ \hline
    0 & 1 & 2000 \\ \hline
    1 & 1 & 1000 \\ \hline
    \end{tabular}
    \caption{\label{fortuites}\'Evolution du nombre de coïncidences fortuites avec la largeur de la fenêtre de détection} 
\end{table}

\end{enumerate}

\subsection{Diode de pompe}
\par Une diode @405nm est utilisée pour créer les paires de photons par un effet non-linéaire.

\subsection{Paires de photons obtenus par conversion paramétrique spontanée}

\subsection{Réglage des deux lentilles de focalisation}
\par Pour détecter le \textit{maximum} de photon doublés, il faut faire l'image de la zone où ils se forment sur les détecteurs.

\begin{enumerate}
\setcounter{enumi}{6}
    \item \par D'après la relation de conjugaison de Descartes, on a :
\[ \frac{1}{\bar{z}'} - \frac{1}{\bar{z}} = \frac{1}{f'} \]
    \par Donc 
\[ \bar{z}' = \frac{1}{\frac{1}{f'} + \frac{1}{\bar{z}}} \]
    \par On trouve que la distance entre les lentilles et les détecteurs est de 81mm. Les détecteurs sont en pratique assez bien placés, et on ne peut les ajuster que transversalement.

    \item \par Pour cette conjugaison, le grandissement est donné par $g_y = |\bar{z}' / \bar{z}|$, il vaut numériquement 8\%. La taille de la tâche d'Airy pour cette conjugaison est:

\begin{align*}
    \Phi_{Airy} &=& 1.22 \ \lambda \ \sin \alpha \\
        &=& 1.22 \ \lambda \ \frac{z'}{\Phi_{lentille}/2} \\
        &=& 1.22 \times 0.810 \mu \mathrm{m} \times \frac{81}{12.7/2} \\
        &=& 13 \mu \mathrm{m}
\end{align*}

    \par La dimension de cette tâche d'Airy est très petite comparée à la taille du détecteur ($180 \mu \mathrm{m}$).Ce n'est donc pas la diffraction qui limite l'entrée des photons dans les photodiodes.
    \par En outre la taille minimale de focalisation du "waist" de la diode de pompe dans le cristal doubleur est de l'ordre de la longueur d'onde \footnote{En utilisant $2\,w_0 = \frac{4\lambda}{\pi\, \Theta_{\mathrm{div}}}$ on trouve un waist de $40 \mu \mathrm{m}$ pour $\Theta_{\mathrm{div}}=3\degres/2$. La dimension de l'image de ce waist est petite comparée à la taille du détecteur.}, et comme le grandissement est assez faible, on peut affirmer que si le faisceau de pompe est bien focalisé dans le cristal, alors tous les photons émis dans la bonne direction sont collectés par les détecteurs.

\end{enumerate}

\subsection{Optimisation du nombre de coïncidences}

\par Le doublet de cristal de BBO doit être bien orienté pour optimiser le nombre de coïncidences.

\begin{enumerate}
\setcounter{enumi}{8}
    \item \par La polarisation incidente est verticale. Pour optimiser l'effet non-linéaire, il faut chercher à modifier l'angle entre l'axe optique du cristal et la direction de propagation du faiscau de pompe. L'acceptance angulaire du cristal n'est pas la même suivant les deux directions contenues dans le plan de sa face d'entrée. L'une des vis aura donc plus d'effet sur l'efficacité de conversion que la seconde. De plus une inclinaison du cristal entraîne une déviation du faisceau de sortie. Il est difficile de prédire \textit{a priori} quel sera l'axe qui sera le plus sensible, mais on s'en rend compte rapidement en pratique.
    \item \par Le nombre de coïncidences fortuites est inférieur à 10 coups par seconde. On pourra utiliser le nombre de coïncidences corrigé si on effectue dans la suite des mesures avec analyseurs croisés, en effet dans certaines configurations, le rapport bruit sur signal pourra devenir important.
    \item Le rapport entre le nombre de coïncidences et le nombre de coups individuels est $3/(5093+11583)=1.7 \ 10^{-4}$.
\end{enumerate}

\subsection{Lames demi-onde et cubes séparateurs}

\begin{enumerate}
\setcounter{enumi}{11}
    \item \par Lorsque la lame $\lambda/2$ est verticale, une polarisation incidente $\ket{V}$ ne sera pas modifiée (car elle est parallèle à un axe neutre de la lame), à la sortie du cube séparateur de polarisation, le détecteur obtiendra donc 0 (le photon sera sorti du mauvais côté du cube). Si la polarisation incidente était $\ket{H}$, elle ne sera pas modifiée à la sortie de la lame (car elle est parallèle à l'autre axe neutre de la lame); et on détectera ce photon à la sortie du cube. Dans cette configuration, l'ensemble forme donc un analyseur horizontal.
    \par Si on place la lame à 45\degres, une polarisation incidente $\ket{V}$ sera symétrisée par rapport à cette direction de la lame, elle donnera donc en sortie une polarisation $\ket{H}$. De même une polarisation $\ket{H}$ donnera une $\ket{V}$. Les polarisations sont échangées juste avant le cube, l'ensemble forme alors un analyseur vertical.

    \item \par Nous mesurons les taux de coïncidences lorsque les analyseurs sont parallèles, et lorsqu'ils sont croisés (\textit{cf.} tableau \ref{nb-EPR-base1}).
\begin{table}[H]
\centering
    \begin{tabular}{||c|c|c|c||}
    \hline
    Analyseurs & $n_A$ & $n_B$ & coïncidences corrigés \\ \hline \hline
    parallèles & 5023 & 3480 & 260 \\ \hline
    croisés & 5093 & 11583 & 3 \\ \hline
    \end{tabular}
\caption{\label{nb-EPR-base1}Mesure des coïncidences dans le cas d'analyseurs parallèles et croisés pour un EPR (pas encore ajusté) dans la base non diagonale} 
\end{table}

    \par Nous en déduisons les taux de coïncidences dans le tableau \ref{taux-EPR-base1}.

\begin{table}[H]
\centering
    \begin{tabular}{||c|c||}
    \hline
    Analyseurs & Taux de coïncidences \\ \hline \hline
    parallèles & 3,06E-002 \\ \hline
    croisés & 1,80E-004 \\ \hline
    \end{tabular}
\caption{\label{taux-EPR-base1}Calcul des taux de coïncidences dans le cas d'analyseurs parallèles et croisés pour un EPR (pas encore ajusté) dans la base non diagonale} 
\end{table}

\par Le taux de coïncidences est très élevé lorsque les analyseurs sont parallèles, comparé au cas où ils sont croisés. Le contraste de la mesure est donc assez bon --- $(\tau_{//} - \tau_{\perp})/(\tau_{//} + \tau_{\perp}) = 98.8\%$ !
\end{enumerate}


\end{document}



